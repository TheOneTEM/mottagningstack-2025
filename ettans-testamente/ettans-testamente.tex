\documentclass[a4paper, 12pt]{article}
\usepackage[utf8]{inputenc}
\usepackage[T1]{fontenc}

\usepackage[swedish]{babel}

\usepackage{centernot}

\title{$\centernot\exists$ttans Guide Till Efterkör}
\date{}

\begin{document}
\maketitle

Vid fall av högläsning bör detta dokument med fördel läsas i Drifvarröst. Speciellt avsnittet ``$\centernot\exists$ttan\dots''. Resten av dokumentet bör dock tas på allvar, då det innehåller en hel del seriösa råd.

\section{$\centernot\exists$ttan\dots}

Bakom trettio meter betong, fem blyinfattade pansardörrar, och Bolibompadrakens försvunna Ovveoskuld, har den nuvarande TVÅAN observerat den dåtida $\centernot\exists$ttan, när den dåtida $\centernot\exists$ttan försökt sig på att planera ett efterkör. Det, har inte gått så bra. Inte så bra alls. 

Därför, har den nuvarande TVÅAN, i all sin \textit{\textbf{OÄNDERLIGA}} godhet, beslutat sig för att kompilera en hel del instruktioner för framtida $\centernot\exists$ttan just angående efterkörsplanering.

\section{Punkt 1. Efterkörsansvarige}

Efterkörsansvarig, $\centernot\exists$ttan, Det är den person, eller personer, som tar över festlokalen efter att Sittningen, försvunnit ur festlokalen. 

$\centernot\exists$ttans Efterkörsansvarige bör hålla god kontakt med sin efterkörspersonal, och se till att alla får den information som krävs. $\centernot\exists$ttans Efterkörsansvarige bör också se till att hålla möte(n) med Efterkörspersonalen för att hålla dessa uppdaterade kring den rådande situationen. 

$\centernot\exists$ttans Efterkörsansvarige bör också hålla god kontakt med varandra och fördela arbetet väl, så att en person inte behöver göra allt.

\section{Punkt 2. Drinkar}

$\centernot\exists$ttan bör inte vara \textit{alltför} ambitiös med sina drinkar. Detta är för att en signifikant del av alkoholen beställs hos \textit{\textbf{DKM}}, som inte är så villiga att beställa ingedienser av den mer sällsynta sorten. Dock, uppmanas $\centernot\exists$ttan att vara kreativa. Inte bara med recepten, utan även med namngivningen. I Mottagningstacket 2025 döpte vi till exempel en drink till "Drakens Eld", för att det var Bolibompatema.

$\centernot\exists$ttan bör också vara klara med drinkmenyn i god tid, då \textit{\textbf{DKM}} beställer på måndagar.

En sak som är nämnvärd just angående drinkar, är att $\centernot\exists$ttan \textbf{inte får ha mer än 4 cl sprit i en drink. Det, är inte bara cringe. Det är dessutom olagligt.}$\centernot\exists$ttan gör bäst i att komma ihåg det. \textbf{Nötter är också förbjudna i META.} $\centernot\exists$ttan gör också bäst i att komma ihåg det.

Om $\centernot\exists$ttan mot all förmodan behöver köpa in egna ingredienser, uppskatta då att behöva göra 1 drink per person och timme. För en fest med 100 pers på 6 timmar, räkna då med 600 drinkar, där en majoritet av dessa är alkoholhaltiga.

\section{Punkt 3. Låtar}

$\centernot\exists$ttan bör kompilera en spellista som är något längre än efterkörets maximala längd, så att $\centernot\exists$ttan inte tar slut på låtar att spela.

DJ bord finns i META men den nuvarande TVÅAN har uppskattat att det skulle ta för lång tid att förbereda alla låter för DJing. Om det $\exists$ någon $\centernot\exists$ttan som jättegärna vill DJ:a, så är det lowkey fritt fram tho. Men Spotify räcker.

Det är inte så kul att jobba till låtar man inte kan, så de $\centernot\exists$ttan som är efterkörsansvariga kan även ta hjälp av den personal på Efterköret, samt övrig personal på Sittningen, inklusive Konglig Öfverdrif, för att kompilera spellistan. Då har alla minst en låt med i listan som hen tycker är bra.

\section{Punkt 4. Efterkörspersonalen.}

Efterkörspersonalen kan förslagsvis delas in i följande kategorier:

\begin{enumerate}
    \item GuardMasters, eller GM, eller GrandMothers.\begin{itemize}
        \item Dessa är Dörrvakter. Minst 2, så det alltid är någon som håller koll på dörren.
        \item Håller koll på dörren och ser till att det inte är för mycket folk eller fel personer i lokalen.
    \end{itemize}
    \item DrinkAndDiskMasters, eller DM
\end{enumerate}


\section{Punk 5. Nödläge.}

Ring polisen för i helvete. 

\end{document}